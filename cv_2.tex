%%%%%%%%%%%%%%%%%%%%%%%%%%%%%%%%%%%%%%%%%%%%%%%%%%%%%%%%%%%%%%%%%%%%%%%%%%%%%%%%
% Original author:
% Rensselaer Polytechnic Institute 
% (http://www.rpi.edu/dept/arc/training/latex/resumes/)
%
% Modified by:
% Daniel L Marks <xleafr@gmail.com> 3/28/2015
%%%%%%%%%%%%%%%%%%%%%%%%%%%%%%%%%%%%%%%%%%%%%%%%%%%%%%%%%%%%%%%%%%%%%%%%%%%%%%%%

%-------------------------------------------------------------------------------
%	PACKAGES AND OTHER DOCUMENT CONFIGURATIONS
%-------------------------------------------------------------------------------

%%%%%%%%%%%%%%%%%%%%%%%%%%%%%%%%%%%%%%%%%%%%%%%%%%%%%%%%%%%%%%%%%%%%%%%%%%%%%%%%
% You can have multiple style options the legal options ones are:
%
%   centered:	the name and address are centered at the top of the page 
%				(default)
%
%   line:		the name is the left with a horizontal line then the address to
%				the right
%
%   overlapped:	the section titles overlap the body text (default)
%
%   margin:		the section titles are to the left of the body text
%		
%   11pt:		use 11 point fonts instead of 10 point fonts
%
%   12pt:		use 12 point fonts instead of 10 point fonts
%
%%%%%%%%%%%%%%%%%%%%%%%%%%%%%%%%%%%%%%%%%%%%%%%%%%%%%%%%%%%%%%%%%%%%%%%%%%%%%%%%

\documentclass[margin]{res}  

% Default font is the helvetica postscript font
\usepackage{helvet}
\usepackage{hyperref}

% Increase text height
\textheight=670pt

\begin{document}

%-------------------------------------------------------------------------------
%	NAME AND ADDRESS SECTION
%-------------------------------------------------------------------------------
\name{GOURAV KHULLAR}

% Note that addresses can be used for other contact information:
% -phone numbers
% -email addresses
% -linked-in profile

\address{ERC 450, Kavli Institute for Cosmological Physics(KICP)\\William Eckhardt Research Center, The University of Chicago\\5640 S Ellis Ave.\\Chicago, IL, 60637, USA}
\address{gkhullar@uchicago.edu\\+1 331 980 9560\\\href{http://kicp.uchicago.edu/people/profile/gourav_khullar.html}{Webpage: KICP}\\\href{https://github.com/gkhullar}{Github: gkhullar} }

% Uncomment to add a third address
%\address{Address 3 line 1\\Address 3 line 2\\Address 3 line 3}
%-------------------------------------------------------------------------------
\begin{resume}

%	EDUCATION SECTION
%-------------------------------------------------------------------------------
\section{RESEARCH INTERESTS}
{\bf Formation and evolution} of galaxies in massive {\bf galaxy clusters.}\\
Physics of galaxies in the {\bf high-redshift universe} using strong gravitational lensing.\\
{\bf Spectrophotometric characterization} of galaxy cluster environments via {\bf multi- wavelength observations}: X-ray, UV, optical and infrared data from Chandra X-ray Observatory, Hubble Space Telescope, and Magellan Telescopes.\\
{\bf Statistics} of strong and weak gravitational lensing fields and arcs around galaxy clusters.

%%%%%%%%%%%%%%%%%%%%%%%%%%%%%%%%%%%%%
%%%%%%%%%%%%%%%%%%%%%%%%%%%%%%%%%%%%%
\section{EDUCATION}
\textbf{PhD Candidate, Dept. of Astronomy and Astrophysics, Kavli Institute for Cosmological Physics (KICP), University of Chicago}, Chicago, IL, USA (2015-present)\\
Thesis Advisor: Prof. Michael D. Gladders; Galaxy Clusters and Gravitational Lensing.

\textbf{Masters of Advanced Study (MASt.) in Astrophysics, Institute of Astronomy (IoA), University of Cambridge}, Cambridge, UK (2014-15) \hfill 
Grade: 2:1\\
INLAKS Foundation Fellow at Fitzwilliam College, University of Cambridge\\
Thesis: \textbf{AGN and Galaxy properties in the Dark Energy Survey} (Prof. Richard McMahon, IoA)\\
Analysis of multi-frequency data to characterise spectral energy properties of active galaxies. Visual classification of galaxy merger scenarios in deep field survey data from DES, with comparison from CANDELS (Hubble) and WISE data in the COSMOS field.

\textbf{Erasmus Mundus Undergraduate Exchange Semester, Aalto University}, Helsinki, Finland (2014)\\
Erasmus Fellow, Dept. of Physics, Aalto University

\textbf{Bachelor of Technology (B.Tech.), Engineering Physics, Indian Institute of Technology Delhi (IITD)}, New Delhi, India (2010-14) \hfill GPA:8.7/10\\
Thesis: \textbf{Stellar Speckle Interferometry and Adaptive Optics} (Prof. Kedar Khare, Dept. of Physics)\\
Experimental analysis of Fourier speckle patterns to attain diffraction-limited resolutions and development of a Shack-Hartmann Wavefront reconstruction system. Simulation of turbulent atmospheres through algorithmic development of wavefront profiles.

%%%%%%%%%%%%%%%%%%%%%%%%%%%%%%%%%%%%%
%%%%%%%%%%%%%%%%%%%%%%%%%%%%%%%%%%%%%
\section{PROFESSIONAL EMPLOYMENT}

\textbf{Teacher and Instructor}, Physics of Stars, The College, University of Chicago \hfill 2018 \\
\textbf{Researcher}, Aalto University Metsahovi Radio Observatory, Helsinki, Finland \hfill 2014\\
Advisor: Dr. Joni Tammi (Director, Metsahovi Observatory)\\
\textbf{Summer Researcher}, University of Victoria, and Canadian Computational Cosmology Collaboration (C4), Victoria, Canada \hfill 2013\\
Advisors: Prof. Arif Babul (Director,C4) and Dr. Fabrice Durier\\
\textbf{Researcher} Indian Institute of Astrophysics (IIA), Bangalore, India \hfill 2012\\
Advisor: Prof. G.C. Anupama (Lead Scientist, Thirty Meter Telescope)\\
\textbf{Summer Researcher}, University of Oxford, Oxford, UK, \hfill 2012\\
Advisor: Prof. Philipp Podsiadlowski

%%%%%%%%%%%%%%%%%%%%%%%%%%%%%%%%%%%%%
%%%%%%%%%%%%%%%%%%%%%%%%%%%%%%%%%%%%%
\section{AWARDS}
Graduate Student Leadership Award, University of Chicago \hfill 2018\\
Inclusive Pedagogy Grant, Chicago Center for Teaching, University of Chicago \hfill 2018\\
-Diversity, Equity and Inclusion Journal Club (DEIJC), Astronomy \& Astrophysics\\
Inclusive Pedagogy Grant, Chicago Center for Teaching, University of Chicago \hfill 2018\\
-Physics of Stars, The College, University of Chicago\\
AAS Education and Professional Development Grant, Astrobites Education \hfill 2017,2018\\
Brinson Fellowship for Summer Graduate Research, University of Chicago \hfill 2016,2017\\
Jerry Rao Fellowship for Graduate School, University of Chicago \hfill 2015-16\\
Finalist, Graduate Teaching Assistant Award, University of Chicago \hfill 2015-16\\
INLAKS Foundation Scholarship, University of Cambridge, UK \hfill 2014-15\\
Erasmus Mundus Scholarship, Aalto University, Finland \hfill 2013-14\\
Outstanding Contribution Award, Co-Curricular and Academic Activities \hfill 2013 \\
Indian Institute of Technology Delhi, India\\
Research Student Scholarship, Indian Institute of Astrophysics, India \hfill 2012\\
Semester Excellence Awards for Three Semesters, IIT Delhi \hfill 2011-13\\
KVPY Scholarship for undergraduates, Government of India (declined )\hfill 2010\\

%%%%%%%%%%%%%%%%%%%%%%%%%%%%%%%%%%%%%
%%%%%%%%%%%%%%%%%%%%%%%%%%%%%%%%%%%%%
\section{PUBLICATIONS}

"Spectroscopic Confirmation of Five Galaxy Clusters at z > 1.25 in the 2500 deg2 SPT-SZ Survey", Khullar et al., The Astrophysical Journal, Volume 870, Issue 1, article id. 7, 16 pp. (2019)\\
"Galaxy populations in the most distant SPT-SZ clusters - I. Environmental quenching in massive clusters at $1.4-1.7$", arXiv:1807.09768 (2018)\\
"Origin of a massive hyper-runaway subgiant star LAMOST-HVS1 -- Implication from Gaia and follow-up spectroscopy", ApJ 873, 116 (2019)\\
"A Detailed Study of the Most Relaxed SPT-selected Galaxy Clusters: Properties of the Cool Core and Central Galaxy", The Astrophysical Journal, Volume 870, Issue 2, article id. 85, 13 pp. (2019)\\
"The Dark Energy Survey: Data Release 1", The Astrophysical Journal Supplement Series, Volume 239, Issue 2, article id. 18, 25 pp. (2018)\\
"Cluster Cosmology Constraints from the 2500 deg$^2$ SPT-SZ Survey: Inclusion of Weak Gravitational Lensing Data from Magellan and the Hubble Space Telescope", arXiv:1812.01679 (2018)\\
"2016 QU89", Minor Planet Electronic Circ., No. 2018-V144 (2018) (2018)\\
"2016 QV89", Minor Planet Electronic Circ., No. 2018-V143 (2018) (2018)\\
"X-ray Properties of SPT Selected Galaxy Clusters at 0.2<z<1.5 Observed with XMM-Newton", arXiv:1807.02556 (2018)\\

%%%%%%%%%%%%%%%%%%%%%%%%%%%%%%%%%%%%%
%%%%%%%%%%%%%%%%%%%%%%%%%%%%%%%%%%%%%
\section{OBSERVING EXPERIENCE AND PROPOSALS}

Magellan/Clay: LDSS3 Spectroscopic observations of Strong Lenses observed via SPT (2 night) \hfill June 2019\\
Magellan/Clay: LDSS3 Spectroscopic observations of high-redshift galaxy clusters observed via SPT (2 night) \hfill June 2019\\
Magellan/Clay: LDSS3 Spectroscopic observations of Strong Lenses observed via SPT (2 nights) \hfill Dec 2018\\
Magellan/Clay: LDSS3 Spectroscopic observations of Strong Lenses observed via SPT (3 nights) \hfill Jan 2018\\
Magellan/Clay: MIKE Spectroscopic observations of Hyper-velocity stars \hfill Jan 2018\\
Magellan/Clay: LDSS3 Spectroscopic observations of high-redshift galaxy clusters observed via Dark Energy Survey and Spitzer archival data (2 nights) \hfill Nov 2017\\
CTIO/Blanco: Dark Energy Survey Year 5 Operations and Observations (7 nights) \hfill Oct 2017\\
Magellan/Clay: LDSS3 Spectroscopic observations of SPT-discovered strong lenses (2 nights) \hfill June 2016\\
Magellan/Clay: LDSS3 Spectroscopic observations of high-redshift galaxy clusters discovered via SPT (2 nights) \hfill Jan 2016\\
Himalayan Chandra Telescope, Indian Insitute of Astrophysics: Photometric observations of Type Ib supernovae (1 night) \hfill Dec 2012\\

Proposal I: Magellan/Clay: PI, PISCO \textit{griz} photometric observations of high-redshift SPT galaxy clusters (observed Nov 2016)\\
Proposal II: Magellan/Clay: Co-PI, LDSS3 spectroscopic observations of high-redshift DES galaxy clusters (observed Nov 2017)\\
Proposal III: Co-I, HST Proposal 15307, "Building the SPT-HST Legacy: Imaging Massive Clusters to z=1.5" (observed August 2018-present)\\
HST Cycle 27 Proposals\\

Observational design, data reduction and analysis:\\
Space Based Observatories: Hubble Space Telescope (ACS, WFC3), Spitzer Space Telescope (IRAC), Chandra X-ray Observatory (ACIS)\\
Ground Based Observatories: Magellan Telescopes (LDSS3, MIKE, PISCO)\\
JWST Early Release Science Proposal (PI: J. Rigby, NASA Goddard)

%%%%%%%%%%%%%%%%%%%%%%%%%%%%%%%%%%%%%
%%%%%%%%%%%%%%%%%%%%%%%%%%%%%%%%%%%%%
\section{INVITED AND CONTRIBUTED TALKS AND POSTERS}
"Spectroscopic Confirmation of Five Galaxy Clusters at z $>$ 1.25 in the 2500 deg$^2$ SPT-SZ Survey", American Astronomical Society, 232nd AAS Meeting, id. 324.05 (Jun 2018)\\
"Astrobites as a Pedagogical Tool in Classrooms", American Astronomical Society, 232nd AAS Meeting, id. 209.01 (Jun 2018)\\
"Diversity Equity and Inclusion Journal Club", SACNAS MidWest Meeting, University of Chicago (May 2018)\\
"Stellar Population Synthesis analysis of member galaxies in SPT-discovered high-redshift Clusters", SPT Clusters Collaboration Meeting (April 2018)\\
"Spectroscopic follow-up of high-redshift galaxy clusters", Dark Energy Survey Collaboration Meeting, Chicago (June 2017)\\
"2pt correlation function distortions and mass reconstruction of galaxy clusters", Dark Energy Survey Collaboration Meeting, Chicago (June 2017)\\
"Still Red and Dead? Measuring feedback and star-formation in clusters at z $>$ 1", American Astronomical Society, 228th AAS Meeting, (June 2016)\\
`AGN and Host Galaxies in the Dark Energy Survey', Fitzwilliam Graduate Conference, University of Cambridge (May 2015)\\
'Spectroscopic Analysis of Type Ib Supernova SN2004ao', Tryst, IIT Delhi, India (March 2013)\\ 
'H1 21cm analysis of nebular clouds', Radio Astronomy Winter School, Inter-University Centre for Astronomy and Astrophysics (IUCAA), India (December 2011)\\
%%%%%%%%%%%%%%%%%%%%%%%%%%%%%%%%%%%%%
%%%%%%%%%%%%%%%%%%%%%%%%%%%%%%%%%%%%%

\section{WORKSHOPS AND MEETINGS}

.AstroX Conference, STScI, Baltimore, September 2018\\
Data Visualization and Exploration in the LSST Era Workshop, NCSA, Urbana-Champaign, August 2018\\
SACNAS Midwest Meeting, University of Chicago, May 2018\\
Using Python to Search NASA's Astrophysics Archives (Remote), IPAC, June 2018\\
ALMA Proposal Workshop, Northwestern University and NRAO, March 2018\\
ComSciCon Chicago Science Outreach Workshop, August 2017\\
SPT Cluster Collaboration Meetings, Argonne, Chicago, 2017,2018\\
Future Cosmic Surveys Workshop, University of Chicago, Oct 2016\\
CMB-S4 Meeting, University of Chicago, Oct 2016\\
Cosmology Using Low Resolution Spectroscopy in 2020s, University of Chicago, 2016\\
Dark Energy Survey Chicagoland Collaboration Meetings, 2015,2016,2017\\
World Wide Telescope Developer Workshop, Nov 2015\\
AstroStatistics Workshop, Royal Statistical Society, London, Dec 2014\\
Workshop on Cosmology, Inter-University Centre for Astronomy and Astrophysics, India Nov 2012
%%%%%%%%%%%%%%%%%%%%%%%%%%%%%%%%%%%%%
%%%%%%%%%%%%%%%%%%%%%%%%%%%%%%%%%%%%%

\section{OUTREACH AND PROFESSIONAL SERVICE}
Founder, Diversity, Equity and Inclusion Journal Club (DEIJC), Astronomy and Astrophysics - KICP, University of Chicago \hfill 2017-present\\
Media Intern, 232nd AAS Meeting, American Astronomical Society \hfill Jun 2018\\
Graduate Mentor, Dept of Astronomy and Astrophysics, University of Chicago \hfill 2018-19\\
IMPACT Mentor, International Student Mentorship Program, University of Chicago \hfill 2016-18\\
Instructor and Course planner, Space Explorers Summer Institutes for underprivileged high school students near University of Chicago \hfill 2016,2017,2018\\
University of Chicago Committees: Dean's Student Advisory Committee (2015-17), Equity and Inclusion Council(2018-present)\\
Astrobites, the astro-ph reader's digest: Author (2016-2018), Education Researcher (2018-present), Co-Administrator (2018-present)\\
Teaching Committee and Student Representative, Part III and MASt Astrophysics, Institute of Astronomy, University of Cambridge \hfill 2014-15\\
Academic Affairs Officer, Fitzwilliam College, University of Cambridge \hfill 2014-15\\
President, Astronomy Club, Indian Institute of Technology Delhi \hfill 2012-13\\
Physics Representative, Academic Committee, Indian Institute of Technology Delhi \hfill 2012-13

Mentoring Undergraduate students in Gladders Research Group, University of Chicago:\\
1. Katya Gozman, Sophomore, University of Chicago (SED fiting and Stellar Population Synthesis for high-redshift galaxies)\\
2. Sebastian Fernandez-Mulligan, Senior Undergraduate, University of Chicago (MCMC algorithms and statistical methods for astrophysical simulations)

Conflict Resolution Training, November 2018\\
Bystander Intervention and Support Training, February 2019


\section{PRESS}

\href{https://news.umich.edu/u-m-researchers-confirm-massive-hyper-runaway-star-ejected-from-the-milky-way-disk/}{Massive Hyper-Runaway Star Ejected from Milky Way's Disk} \hfill 2019\\
\href{http://kicp.uchicago.edu/events/kicp_yerkes.html#id_959}{The Physics of Toys, Yerkes Summer Institute, KICP, University of Chicago} \hfill 2017\\
\href{http://kicp.uchicago.edu/events/kicp_yerkes.html#id_785}{Spy vs. Spy, Yerkes Summer Institute, KICP, University of Chicago} \hfill 2016\\









% \par
% \textbf{Project One}: 
% Lorem ipsum dolor sit amet, consectetur adipiscing elit. Pellentesque semper 
% vulputate vestibulum. Pellentesque viverra orci vitae dolor ultrices, eu semper 
% ex blandit. 

% \par
% \textbf{Project Two}:
% Nunc quis diam non sem tempus ornare. Nullam eleifend ligula varius 
% elementum pharetra. Donec fringilla quis mauris quis tristique. Vestibulum ac 
% accumsan quam, a semper mauris. 
% \par
% \textbf{Project Three}: 
% Pellentesque urna odio, euismod id nibh ut,  lobortis eleifend odio. Sed 
% placerat arcu pulvinar dictum egestas. Vivamus  scelerisque commodo urna sit 
% amet imperdiet. Fusce ac sodales quam.

% \par
% \textbf{Project Four}: 
% Integer mattis erat nec tortor fermentum mollis. Donec sit amet justo vitae 
% tellus pharetra tempor et nec metus. Nam et consequat dui, convallis laoreet 
% libero. 

% \par
% \textbf{Project Five}: 
% Phasellus eu imperdiet purus. Maecenas id purus leo. Nulla quis purus varius, 
% imperdiet mauris at, pellentesque erat. 

%-------------------------------------------------------------------------------
%-------------------------------------------------------------------------------
%	EXPERIENCE SECTION
%-------------------------------------------------------------------------------
% % Modify the format of each position
% \begin{format}
% \title{l}\employer{r}\\
% \dates{l}\location{r}\\
% \body\\
% \end{format}
% %-------------------------------------------------------------------------------

% \section{EXPERIENCE}
% \employer{Employer One}
% \location{Location One}
% \dates{Dates One}
% \title{\textbf{Title One}}
% \begin{position}
% Vivamus ullamcorper a lacus non laoreet. Etiam ultricies in est quis finibus. 
% Curabitur posuere est quis felis sodales, vitae aliquam ex consequat. In non 
% magna felis.
% \end{position}

% \employer{Employer One}
% \location{Location One}
% \dates{Dates One}
% \title{\textbf{Title One}}
% \begin{position}
% Donec venenatis volutpat tortor, quis fringilla turpis ornare et. Curabitur 
% neque enim, facilisis vitae lacus non, malesuada volutpat tortor. Etiam 
% elementum neque nibh, ac faucibus ligula tempor vel. Pellentesque a pharetra 
% neque, vel facilisis nunc.
% \end{position}

% \employer{Employer One}
% \location{Location One}
% \dates{Dates One}
% \title{\textbf{Title One}}
% \begin{position}
% Mauris lacinia tellus vel elit lobortis, ut convallis tellus euismod. Praesent 
% sagittis magna non nisl rutrum, dapibus mollis ante commodo. Donec scelerisque 
% velit a consequat vehicula. Nam eget dignissim est, et varius arcu. Vivamus eu 
% lacus feugiat, ullamcorper odio ut, blandit enim. Suspendisse potenti. 
% Vestibulum auctor purus et massa luctus, quis cursus mauris laoreet. Praesent 
% luctus dictum justo vitae dictum.
% \end{position}
%-------------------------------------------------------------------------------
%	Interests
% %-------------------------------------------------------------------------------
% \section{INTERESTS}
% Interest One, Interest Two, Interest Three, Interest Four, ...Et Cetera.
%-------------------------------------------------------------------------------
\end{resume}
\end{document}